\chapter{Appendix}

\begin{figure}[hbt]
    \centering
    \includestandalone[width=1.0\textwidth]{../figures/overfitting1}
    \caption[Under and Overfitting]{Comparison of under- and overfitting (left and right) to a balanced well generalizing model (center). \cite{Thoma.2015}}
    \label{fig:overfitting1}
\end{figure}

\begin{figure}[H]
    \centering
    \includestandalone[width=1\textwidth]{../figures/crossvalidation}
    \caption[Cross-validation]{Subdivision of known data for training (test data excluded). The validation data (blue) are used for error estimation while training with the training set (white).  For each step there is a model trained and tested. \cite{Skillmon.2018}}
    \label{fig:crossvalidation}
\end{figure}

\begin{figure}[hbt]
\centering
\begin{subfigure}[b]{0.5\textwidth}
    \centering
    \includestandalone[width=1.0\textwidth]{../figures/gradient_descent2}
    \caption[Small Learning Rate]{Small Learning Rate}
    \label{fig:gradientdescent1}
\end{subfigure}
\begin{subfigure}[b]{0.5\textwidth}
    \centering
    \includestandalone[width=1.0\textwidth]{../figures/gradient_descent1}
    \caption[Medium Learning Rate]{Medium Learning Rate}
    \label{fig:gradientdescent2}
\end{subfigure}
\begin{subfigure}[b]{0.5\textwidth}
    \centering
    \includestandalone[width=1.0\textwidth]{../figures/gradient_descent3}
    \caption[High Learning Rate]{High Learning Rate}
    \label{fig:gradientdescent3}
\end{subfigure}
\caption[Learning Rate Comparison]{Comparison of different initial learning rates for training machine learning models. Plotting with function gradient and local minimum in the center (yellow). \cite{user30471.2020}}
\label{fig:gradientdescent}
\end{figure}