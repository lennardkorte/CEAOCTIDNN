\chapter{Abstract}

\Gls{atherosclerosis} is one of the leading but preventable and treatable causes of death worldwide. For effective treatment, e.g. choice of stent size and its positioning, it is essential for the attending physician to have an accurate idea of the intravascular anatomy. Lesions (plaque) and arterial walls can be visualized with \acrfull{ivoct}, a high-resolution imaging technique, that can be applied by experts. Due to the high number of images captured during clinical routines, research spends tremendous efforts on automatic plaque detection (image classification) with \acrfull{dl} in polar as well as cartesian representation for fast and accurate decision support. \Acrfull{da} is an essential part of training \acrshort{dl} models in machine learning. A variety of augmentation strategies, including cropping, flipping, and convolutional filters, have been proposed. They have shown to maintain important characteristics of \acrshort{ivoct} images without handcrafting features, especially when in cartesian representation. While data augmentation has been commonly used for deep learning in \acrshort{ivoct} imaging, little work has been done to determine which augmentation techniques capture medical image statistics best and lead to a higher model performance. This work provides an overview about conventional techniques and proposes new, advanced techniques tested on a data set consisting of labeled in vivo patient images. Furthermore, augmentation techniques are compared and those that add value are proposed for further investigation building domain specific training strategies. Finally, it was shown that the models performance is determined by the extent to which relevant information and features of the original medical images are preserved in the augmented training sets. Overall, we find that a model trained with data that are augmented by the discussed techniques, for instance shifting or shearing, may boost the performance by 0.05 in \acrfull{mcc} and 0.03 in \acrfull{bacc}. On the other hand, less effective transformations such as contrast jittering may deteriorate scores significantly by more than ten percent. Our results show that the right \acrshort{da} strategy is essential for performance optimization when building deep learning-based clinical decision support systems for plaque detection.


