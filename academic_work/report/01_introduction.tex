\chapter{Introduction}

% Hinführung
Coronary heart disease is one of the leading causes of death \cite{HannahRitchie.2018, NationalVitalStatisticsReports.2017} accounting for more that 15\% of all deaths in 2015 \cite{Kolluru.2018}, despite being preventable and treatable. To treat the causes of coronary heart disease, \acrfull{oct} technology can be used to allow a detailed examination of the in vivo tissue anatomy and coronary procurement from within.

% (IV-)OCT
In comparison to the typically used \acrfull{ivus} technology, \acrfull{ivoct} offers a higher micrometer resolution utilizing interferometry low-coherent light \cite{Schippling.2015, Drexler.2008}. Both imaging techniques that can be used for planning stent interventions or during interventions to guide stent placement and characterize atherosclerotic plaque types \cite{Kolluru.2018}. \Acrshort{ivoct} is well suited for embedding in thin medical instruments such as catheters, because the infrared light can be transmitted via glass fibres \cite{HiramG.Bezerra.2009, Gessert.2019}. Due to its low penetration depth capabilities, it only allows visualization of regions directly around the catheter. This is sufficient for the attending physicians to obtain a detailed understanding of the region of interest, i.e. the arterial obstruction. \cite{Gessert.2019} Currently \acrshort{ivoct} is the only imaging technique that enables identification of the most rupture susceptible thin-cap fibroatheroma \cite{GuhaRoy.2016}.

% Why automatically?
A key problem is the large number of frames (often > 500) arising from single 2s to 5s pullbacks in \acrshort{ivoct} interventions \cite{Kolluru.2018}, that cannot be reviewed by the practitioner during clinical routine. Automatic plaque detection is of major importance for a fast and accurate decision support system \cite{Gessert.2019}. These must offer reliable information on the presence and nature of stenosis \cite{Gessert.2019}. Various approaches for automatic \acrshort{ivoct} data analysis have been proposed. Correlating the images to histology data proofs general feasibility of inferring plaque characteristics from \acrshort{ivoct} data \cite{Yabushita.2002}.

% bisherige forschung
Methods for automatic \acrshort{ivoct} data analysis like stent detection \cite{Zhao.2015, Lu.2012} and automatic lumen segmentation \cite{GuhaRoy.2016, Tsantis.2012} have been proposed. Automatic classification approaches using texture and optical properties as features have been proposed for plaque detection and segmentation \cite{Ughi.2013, Celi.2014}. \Acrshort{dl} approaches have been proposed more recently for \acrshort{ivoct} data processing \cite{Abdolmanafi.2017}, in particular \acrfullpl{cnn} have shown remarkable success in image analysis tasks such as segmentation \cite{Long.2015} and classification \cite{Gessert.2018}. Even if plaque detection with \acrshortpl{cnn} has been employed mostly in preliminary studies, they provide promising initial results. Recent standard architectures such as Inception or \acrfullpl{resnet} have shown significant performance improvement for medical image analysis \cite{Long.2015, Gessert.2018}. The approach using \acrshort{cnn}s in medical diagnostic tasks offers the advantage that we can omit the complicated and error-prone step of tissue segmentation. Progress has been made in using \acrshort{dl} for image classification, especially in the prediction and segmentation of cancerous masses in lung, liver or breast scans \cite{Cho.2015, CruzRoa.2013, Fakoor.2013}.

% DL challenges
Nevertheless, various shortcomings still need to be addressed. Polar images are often converted to cartesian representation for easier and more intuitive interpretation by practitioners. One of the questions that remains unanswered is whether either or both images (e.g. in two path \acrfull{dl} architectures) are the more advantageous choice for training the \acrshort{dl} models \cite{Gessert.2018} and relevant for future use. 
The lack of large labeled image data sets from \acrshort{ivoct} interventions is one of the major challenges in the process of creating \acrshort{dl} models that achieve the desired and necessary performance. Nevertheless the effect of having only a limited number of scans can be mitigated by a number of methods to a certain extend, for instance normalization or dropout that prevent the model from \gls{overfitting}. \Acrshort{da} denotes one of the advanced techniques that allow \acrshort{dl} models to extract features from small data sets more effectively.

Hence, the main objective of this bachelor’s thesis was to implement a \acrshort{dl} pipeline to investigate the effects of \acrshort{da} techniques tailored to polar as well as cartesian representation. It was examined which \acrshort{da} techniques have positive effects on the overall model performance, that is, the extent to which augmented data sets retain properties of the original medical images. By conducting a number of experiments we investigate which \acrshort{da} techniques are recommended to use. Their individual as well as combined performance impact on the \acrshort{cnn} has been investigated to form a strategy applicable in the domain of \acrshort{ivoct} imaging. In addition, the question is answered to what extent data augmentation techniques applied on \acrshort{ivoct} data can improve the networks performance. New \acrshort{ivoct} specific \acrshort{da} techniques are presented and included in the analysis. It is assumed that \acrshort{ivoct} images do not vary in their fundamental structure, such that observations can be transferred to other \acrshort{ivoct} data sets. Trained experts labeled the frames of each pullback with binary class characterizations that distinguished between the presence and absence of plaque.
